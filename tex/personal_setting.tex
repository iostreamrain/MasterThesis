\newcommand{\edoctitle}{Master's Thesis} % or Master's Thesis
\newcommand{\degree}{MASTER} % or MASTER
\newcommand{\major}{ENGINEERING} % or SCIENCE
\newcommand{\etitle}{A Study on Inductive Power Transfer Using Coil-Array for Autonomous Underwater Vehicles}     % title of the dissertation
\newcommand{\eauthor}{Yu Cheng} % Author's name
\newcommand{\edate}{\ifcase\month\or
    January\or February\or March\or April\or May\or June\or
    July\or August\or September\or October\or November\or December\fi
    \space\number\day,\space \number\year}
\newcommand{\ekeywords}{Autonomous underwater vehicle, inductive power transfer, underwater wireless power transfer, undersea}

\newcommand{\eabstract}{ % abstract
For a long time, providing a stable, safe, and efficient power supply for underwater electromechanical equipment has always been a concern in deep-sea exploration. Compared with the complicated docking mechanism, potential safety hazards, and expensive price of traditional wet-mate connectors, wireless power transmission (WPT) technology can transmit energy without any electrical contact between the power supply and the electrical equipment, which provides an effective solution to the aforementioned drawbacks of wired charging. There are many uncontrollable factors in the seawater working environment. Therefore, this topic takes the equivalent circuit and magnetic field distribution as the theoretical basis to study the energy transmission characteristics of underwater WPT and proposes corresponding improvements and solutions to the current problems and deficiencies. Especially for the unstable output voltage of the receiver and excessive magnetic flux density at the internal of AUV. This thesis proposed a new type of UWPT system with a coil-array structure, which can reduce the magnetic flux density at the center of the AUV by 40\% compared to the conventional hollow cylindrical structure.
}