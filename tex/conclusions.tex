\chapter{Conclusion}\chaplab{conclusion}

% 在前面章节的探讨中,我们分析了几种常见的wpt系统,并对海水环境中的WPT线圈的参数变化进行了分析。
% 对通过two-ring结构对水下环境的一些属性进行了
% WPT以“电磁电”的传输方式实现了无电气接触的传输,为海下探测仪器与机电设备提供了一种安全可靠的电能输送方案。但是由于在水下,电磁耦合器的耦合系数较低、存在附加电路和介质损耗、WPT的传输效率低等一系列问题,阻碍了该技术在海洋技术领域的普遍应用。本文在课题组前期研究的基础上对非接触式电能传输系统在海洋环境中的电磁辐射,线圈偏移和传输效率进行研究。从系统机理、仿真计算、参数设计、线圈机构设计、系统实现和实验测试等多个角度探讨如何获取较高的传输效率并保持较低的电磁辐射,具有一定的学术价值和现实意义。
% 从实际应用角度出发,充分考虑海洋环境因素,对在不同的谐振补偿结构与负载阻值下的传输效率展开较为详细与深入的理论研究与分析,对研究成果做如下总结:
% 以的互感模型为基碰,提出海水等效阻抗的概念,建立海水环境下的传输模型。通过模型比较,表明海水介质改变了原系统中某些参数的数值,但并未改变空气原空气间隙下系统的传输机理。因此,海水环境下的基本分析方法是:在常规互感模型分析的基础上,结合海水介质对系统内参数的综合影响分析,来指导海水环境下的系统设计。
% 通过对谐振补偿结构和其特征阻抗的设计与计算,讨论了四种补偿结构的机理和适用场合。依据特征阻抗的计算结果,建立了四种补偿结构下的传输效率模型。对应四种补偿结构,具有两类的传输效率模型;次级采用的补偿结构决定了整个系统的传输效率类型。
% 海水环境下的传输效率是其在空气问隙下的传输效率与其在海水介质间隙内共同作用的叠加结果。因此,对海水环境下的传输效率进行分析,可先对空气间隙下的传输效率特性进行计算和分析,然后对海水介质引起的祸流损耗进行计算,最后推广至海水环境下进行分析。对空气间隙下不同谐振补襟结构对应的传输效率特性展开理论分析。分析结果表明,在空气间隙下,两类补偿结构的系统的传输效率具有非常相似的特性。次级串联补偿结构在中小功率场应用更利于传输效率的优化,而次级并联补偿结构在中大功率应用场合更利于传输效率的优化。
% 研究成果为今后进一步开展深入研究与系统开发、完善非接触式传输技术理论、普及非接触式传输技术在海洋领域的应用具有借鉴和实用价值。
WPT uses the electricity-magnetism-electricity transmission mode to realize the transmission without electrical contact, and provides a safe and reliable power transmission scheme for underwater detection instruments and electromechanical equipment. However, due to the low coupling coefficient of the electromagnetic coupler, the existence of additional circuit and dielectric loss, and the low transmission efficiency of WPT, a series of problems hindered the widespread application of this technology in the field of marine technology. In this paper, the electromagnetic radiation, coil offset and transmission efficiency of the non-contact power transmission system in the marine environment are studied on the basis of the preliminary research of the research group. Discussing how to obtain high transmission efficiency and maintain low electromagnetic radiation from multiple perspectives such as system mechanism, simulation calculation, parameter design, coil mechanism design, system implementation and experimental testing, has certain academic value and practical significance.

From the perspective of practical application, with full consideration of marine environmental factors, a more detailed and in-depth theoretical study and analysis of the transmission efficiency under different resonance compensation structures and load resistance values are carried out, and the research results are summarized as follows:

Based on the mutual inductance model, the concept of seawater equivalent impedance is proposed, and the transmission model under seawater environment is established. The comparison of models shows that the seawater medium changes the values of some parameters in the original system, but does not change the transmission mechanism of the system under the original air gap. Therefore, the basic analysis method in the seawater environment is to guide the system design in the seawater environment based on the analysis of the conventional mutual inductance model, combined with the analysis of the comprehensive influence of the seawater medium on the parameters in the system.

Through the design and calculation of the resonance compensation structure and its characteristic impedance, the mechanism and application occasions of the four compensation structures are discussed. According to the calculation results of characteristic impedance, four types of transmission efficiency models under compensation structures are established. Corresponding to the four compensation structures, there are two types of transmission efficiency models; the compensation structure used in the secondary determines the transmission efficiency type of the entire system.

The research results have reference and practical value for further in-depth research and system development, perfecting the theory of non-contact transmission technology, and popularizing the application of non-contact transmission technology in the marine field.
\section{Future works}
It is true that due to time and conditions, the research and work of this article still have further in-depth and perfect places:
% 诚然,由于时间和条件的限制,本文的研究和工作还有进一步深入和完善的地方:
% 充分利用线圏在海水介质内的分布电容,使线圏能产生自谐振,独立构成谐振中继;完善海洋环境下谐振中继与装置的集成与封装,将极大地提高装置的传输效率,并客观上增加装置适用的水下深度。
%  如果我们线圈组的线圈继续变小,我们可以讨论线圈之间正对(正对着的耦合)数量关系来描述传输的性能

% 问题,线圈间的影响还未考虑。

% 平面螺旋线圏
