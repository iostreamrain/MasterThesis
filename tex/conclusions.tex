\chapter{Conclusion}\chaplab{conclusion}

% 在前面章节的探讨中,我们分析了几种常见的wpt系统,并对海水环境中的WPT线圈的参数变化进行了分析。
% 对通过two-ring结构对水下环境的一些属性进行了

WPT uses the electric power to electromagnetic field to electric power transmission mode to realize the transmission without electrical contact, and provides a safe and reliable power transmission scheme for underwater sensors and electromechanical equipments. 
However, under the seawater environment, the electromagnetic coupler has a low coupling coefficient, additional circuit and dielectric loss, and a series of problems such as low transmission efficiency of WPT, which hinder the general application of this technology in the field of marine technology. 
In this thesis, the electromagnetic radiation, coil offset and transmission efficiency of the non-contact power transmission system in the marine environment are studied on the basis of the preliminary research of the research group. Discussing how to obtain high transmission efficiency and maintain low electromagnetic radiation from multiple perspectives such as system mechanism, simulation calculation, parameter design, coil design, system implementation and experimental testing, has certain academic value and practical significance.

From the perspective of practical application, with full consideration of marine environmental factors, a more detailed and in-depth theoretical study and analysis of the transmission efficiency under different resonance compensation structures and load resistance values are carried out, and the research results are summarized as follows:

% 以的空气中WPT模型为基碰,提出海水环境下的线圈的等效模型。
% 通过空气中与水中WPT模型比较,表明海水介质改变了原系统中某些参数的数值,但并未改变空气下系统的无线能量传输原理。
% 因此,海水环境下的基本分析方法是:在常规空气WPT模型分析的基础上,结合海水介质对WPT系统内参数的综合影响分析,来指导海水环境下的WPT系统设计。
Based on the WPT model in the air, an equivalent model of the coils in the seawater environment is proposed.
The comparison between the WPT model in the air and the water shows that the seawater medium changes the values of some parameters in the original system, but does not change the wireless energy transmission principle of the system in the air.
Therefore, the basic analysis method in seawater environment is: on the basis of analysis of conventional WPT model in the air, combined with the analysis of the comprehensive influence of seawater media on the parameters of the WPT system, to guide the design of WPT system under seawater.

% 通过对谐振补偿结构和其特征阻抗的设计与计算,讨论了两种常见的补偿结构的原理以及适用范围。
% 并且分别计算出它们的电源电压与负载电压之间的关系与系统效率。
% 并通过两组不同线圈结构对这两种补偿网络加以验证。
Through the design and calculation of the resonance compensation structure and its characteristic impedance, the principles and scope of application of two common compensation structures are discussed.
And respectively calculate the relationship between their power supply voltage and load voltage and system efficiency.
And through two different coil structures to verify these two compensation networks.

The research results have reference and practical value for further in-depth research and system development, perfecting the theory of non-contact transmission technology, and popularizing the application of non-contact transmission technology in the marine field.

\section{Future works}
Due to the limitation of time and conditions, the research and work of this article have further in-depth and perfect places:

Because of the difficulty of the simulation and the lack of suitable experimental components, this research has not yet optimized the mini-coils in the coil-array. If we can do more research on the size, type, and shape of these coils, we could get better system performance.

Because it is difficult to analyze the mutual inductance between multiple sets of connected coils in a three-dimensional space, we have not yet analyzed the influence between mini-coils.
This makes us wonder how to change the layout of mini-coils to minimize the impact of mini-coils on the UWPT system.
%  如果我们线圈组的线圈继续变小,我们可以讨论线圈之间正对(正对着的耦合)数量关系来描述传输的性能

% 问题,线圈间的影响还未考虑。

% 平面螺旋线圏

% wpt模型,常见补偿网络分析,水下wpt模型
% 水下与空气环境对比分析,
% coilarray结构的提出, 传输效率,电磁辐射
